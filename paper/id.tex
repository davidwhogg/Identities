\documentclass[12pt]{article}
\usepackage{url}

\addtolength{\textheight}{1.5in}
\addtolength{\headheight}{-0.75in}
\sloppy\sloppypar\raggedbottom\frenchspacing

\begin{document}

\section*{Matrix and Gaussian Identities}

{\raggedright
\textbf{Sam~Roweis}%
\footnote{Deceased.
Formerly at the \textsl{Department of Computer Science, New York University}.},
\textbf{David~W.~Hogg}%
\footnote{\textsl{Center for Cosmology and Particle Physics, Department of Physics, New York University}; and the \textsl{Flatiron Institute, a Division of the Simons Foundation}.},
\textbf{Dustin~Lang}%
\footnote{\textsl{Perimeter Institute}.},
\& \textbf{Boris~Leistedt}%
\footnote{\textsl{Center for Cosmology and Particle Physics, Department of Physics, New York University}.}
}

\paragraph{Abstract:}
Across all areas of data analysis, probability, statistics, machine learning, and
indeed a far larger set of domains, the linear algebra of rectangular matrices
is core. And nowhere is this more true than in problems that involve Gaussians (normal
distributions), which appear explicitly or implicitly in many different methods.
Here we assemble a set of mathematical identities and relationships involving
matrices---including scalar, vector, and matrix forms constructed from combinations of
scalars, vectors, and matrices---and their derivatives.
We also assemble a set of identies involving Gaussians.
These sets of identities are not exhaustive,
but rather concentrate on the identities most valuable
for the appearance of matrices and Gaussians in machine-learning and data-analysis contexts.
This paper expands and adds context to some crib sheets that have been
available on the internet for years.

\clearpage
\section{Introduction}

...Comments about the linear algebra they don't teach you in school!

...Comments about the value of Gaussians---which flows from the simplicity of the linear algebra!

...Historical comments about these notes, including when they were written and why.

...Personal note about Sam?

In what follows, we include the Roweis notes verbatim, preserving
the original typesetting decisions as closely as possible.
We have not, however, preserved exactly the equation numbering,
because if we did there would be name collisions between the matrix identities
and the Gaussian identities.
We have also corrected a few tiny typos, including one or two places where
Roweis (deliberately) didn't capitalize letters.
We make these changes with apologies to the faithful and utmost respect to the dead.
For those who want to see the original notes, we have preserved them in a git
repository available online\footnote{\url{https://github.com/davidwhogg/Identities}}.
We follow the verbatim notes with some discussion and some additional forms.

\clearpage
\section{Matrix Identities}

Hello World

\clearpage
\section{Gaussian Identities}

Hello World

\clearpage
\section{Discussion}

Hello World.

\end{document}
